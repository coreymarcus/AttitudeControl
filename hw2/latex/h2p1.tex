\documentclass[]{article}

\usepackage{parskip,amsmath}
\usepackage[margin=0.5in]{geometry}

%opening
\title{Homework 2 Problem 1}
\author{Corey Marcus}

\begin{document}

\maketitle

\newcommand{\CrossProd}[1]{\left[ #1 \times \right]}

\section{Prompt}

Derive the linearized equations of perturbed motion of an axial symmetric satellite spinning around its major axis of inertial $J_2 > J_1 = J3$. Assume a constant perturbation torque acting on the spacecraft.

\section{Answer}

We begin with the equation governing motion:
\begin{equation}
	J \dot{\omega} = -\omega \times J \omega + \tau
\end{equation}
$\omega$ is the rotation rate with respect to inertial in the body frame. $J$ is the inertia matrix in the body frame. $\tau$ is the input torque in the body frame.

We introduce nominal $\bar{\cdot}$ and perturbation $\delta \cdot $ components.
\begin{equation}
	J \left(\dot{\bar{\omega}} + \delta \dot{\omega} \right) = -\left(\bar{\omega} + \delta \omega \right) \times J \left(\bar{\omega} + \delta \omega \right) + \left(\bar{\tau} + \delta \tau \right)
\end{equation}

We will expand the right side.
\begin{align}
	J \left(\dot{\bar{\omega}} + \delta \dot{\omega} \right) & = -\left(\bar{\omega} + \delta \omega \right) \times J \left(\bar{\omega} + \delta \omega \right) + \left(\bar{\tau} + \delta \tau \right) \\
	& = -\left(\bar{\omega} + \delta \omega \right) \times \left(J\bar{\omega} + J \delta \omega \right) + \left(\bar{\tau} + \delta \tau \right) \\
	& = -\left(\bar{\omega} + \delta \omega \right) \times J\bar{\omega} -\left(\bar{\omega} + \delta \omega \right) \times J \delta \omega + \left(\bar{\tau} + \delta \tau \right) \\
	& = -\left(\bar{\omega}  \times J\bar{\omega} + \delta \omega  \times J\bar{\omega} \right) -\left(\bar{\omega} \times J \delta \omega + \delta \omega \times J \delta \omega \right)  + \left(\bar{\tau} + \delta \tau \right) \\
	& = -\left(\bar{\omega}  \times J\bar{\omega} + \delta \omega  \times J\bar{\omega} \right) -\left(\bar{\omega} \times J \delta \omega + \delta \omega \times J \delta \omega \right)  + \left(\bar{\tau} + \delta \tau \right)
\end{align}
We can eliminate the nominal's evolution with the governing equation.
\begin{align}
	 & = -\left( \delta \omega  \times J\bar{\omega} \right) -\left(\bar{\omega} \times J \delta \omega + \delta \omega \times J \delta \omega \right)  + \delta \tau \\
	& = J \bar{\omega} \times \delta \omega -\left(\bar{\omega} \times J \delta \omega + \delta \omega \times J \delta \omega \right)  + \delta \tau \\
	& = \CrossProd{J \bar{\omega}} \delta \omega -\left(\CrossProd{\bar{\omega}} J \delta \omega + \delta \omega \times J \delta \omega \right)  + \delta \tau \\
	& = \left( \CrossProd{J \bar{\omega}} - \CrossProd{\bar{\omega}} J \right) \delta \omega + \delta \omega \times J \delta \omega  + \delta \tau
\end{align}

Then we linearize the peturbations about the nominal, ($\delta \omega = 0$).
\begin{align}
	J \delta \dot{\omega} & \approx \left. \frac{d J \delta \dot{\omega}}{d \delta \omega} \right|_{\delta \omega = 0} \delta \omega  + \delta \tau \\
	& \approx \left( \CrossProd{J \bar{\omega}} - \CrossProd{\bar{\omega}} J \right) \delta \omega + \delta \tau
\end{align}

I don't know how to linearize $\delta \omega \times J \delta \omega$ but matlab symbolic toolbox tells me that when it is linearized and evaluated at $\delta \omega = 0$, it becomes zero.

We will now drop the approximate sign for convenience. But note that the approximation is still present.
\begin{align}
	\delta \dot{\omega} & = \begin{bmatrix}
	J_1^{-1} & 0 & 0 \\
	0 & J_2^{-1} & 0 \\
	0 & 0 & J_3^{-1}
	\end{bmatrix} \left( \CrossProd{J \bar{\omega}} - \CrossProd{\bar{\omega}} J \right) \delta \omega + \delta \tau \\
	& = \begin{bmatrix}
	0 & \frac{\bar{\omega}_3(J_2 - J_3)}{J_1} & \frac{\bar{\omega}_2(J_2 - J_3)}{J_1} \\
	-\frac{\bar{\omega}_3(J_1 - J_3)}{J_2} & 0 & -\frac{\bar{\omega}_1(J_1 - J_3)}{J_2} \\
	\frac{\bar{\omega}_2(J_1 - J_2)}{J_3} & \frac{\bar{\omega}_1(J_1 - J_2)}{J_3} & 0
	\end{bmatrix} \delta \omega + \delta \tau
\end{align}
At this point, note two things; $J_1 = J_3$ and $\bar{\omega}_1 = \bar{\omega}_3 = 0$.
\begin{align}
\delta \dot{\omega} & = \begin{bmatrix}
0 & 0 & \frac{\bar{\omega}_2(J_2 - J_3)}{J_1} \\
0 & 0 & 0 \\
\frac{\bar{\omega}_2(J_1 - J_2)}{J_3} & 0 & 0
\end{bmatrix} \delta \omega + \delta \tau \\
\begin{bmatrix}
\delta \dot{\omega}_1 \\
\delta \dot{\omega}_2 \\
\delta \dot{\omega}_3
\end{bmatrix} & = \begin{bmatrix}
\frac{\bar{\omega}_2(J_2 - J_3)}{J_1} \delta \omega_3 \\
0 \\
\frac{\bar{\omega}_2(J_1 - J_2)}{J_3} \delta \omega_1
\end{bmatrix} + \delta \tau
\end{align}

We can define $k_1 = \frac{J_2 - J_3}{J_1}$ and $k_3 = \frac{J_1 - J_2}{J_3}$. Then the free response of $\delta \omega$ is given as the following.
\begin{align}
	\delta \omega_1 (t) & = \delta \omega_1(0) \cos(\Omega t) + \frac{\delta \dot{\omega}_1 (0)}{\Omega} \sin(\Omega t) \\
	\delta \omega_3 (t) & = \delta \omega_3(0) \cos(\Omega t) + \frac{\delta \dot{\omega}_3 (0)}{\Omega} \sin(\Omega t) \\
	\Omega^2 & = k_1 k_3 \bar{\omega}_2^2
\end{align}

We can make a clever substitution.
\begin{align}
	\frac{\delta \dot{\omega}_1 (0)}{\Omega} & = \frac{\bar{\omega}_2 k_1 \delta \omega_3(0)}{\sqrt{k_1 k_3} \bar{\omega}_2} \\
	& = \sqrt{\frac{k_1}{k_3}}\delta \omega_3(0) \\
	\frac{\delta \dot{\omega}_3 (0)}{\Omega} & = \frac{\bar{\omega}_2 k_3 \delta \omega_1(0)}{\sqrt{k_1 k_3} \bar{\omega}_2} \\
	& = \sqrt{\frac{k_3}{k_1}}\delta \omega_1(0)
\end{align}

And now we have an STM.
\begin{equation}
	\begin{bmatrix}
	\delta \omega_1 (t) \\
	\delta \omega_3 (t)
	\end{bmatrix} = \begin{bmatrix}
	\cos(\Omega t) & \sqrt{\frac{k_1}{k_3}} \sin(\Omega t) \\
	\sqrt{\frac{k_3}{k_1}} \sin(\Omega t) & \cos(\Omega t) \\
	\end{bmatrix} \begin{bmatrix}
	\delta \omega_1 (0) \\
	\delta \omega_3 (0)
	\end{bmatrix}
\end{equation}

We know we have a constant input torque and the general solution for $\delta \omega$ is provided by the following. Some steps have been skipped, this is not the general solution. Will need to come back later.
\begin{align}
	\delta \omega (t) & = \int_{0}^{t} \Phi(t,\sigma) \delta \tau d \sigma
\end{align}

\end{document}
